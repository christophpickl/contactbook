\section{Jargon}\label{sec:jargon}

The following chapter is a rather class specific one, as the language and images developed in the scene ``I grew up in'' are very local and not universal at all.
Yet, they might inspire others, or at least provide some humorist benefit.

\subsection{Vocabulary}\label{subsec:vocabulary}

Sometimes concept are too complex, too difficult to verbalize them fully in their precise definition; or simply there have been no proper words established yet in our (English) language, which made it necessary to develop our own few words to convey a certain meaning efficiently:

\begin{itemize}
    \item \textbf{umpf}: The preferred quality of contact between two bodies which is characterized by properly sharing weight; touch is established throughout the whole body (-part) and there is a considerable amount of pressure; e.g.\ when both partners are in table top position and they feel firmly each other's groundedness.
    \item \textbf{wee}: A scream usually expressed only in moments of heightened levels of alertness/fear, to down-regulate one's own nervous system; also to simply express joy in the moment about a movement; e.g.\ when performing more risky lifts.
    \item \textbf{botsen}: A conflict which arises when the teacher's instructions lead to a resistance based on an internal wisdom; when the ``inner teacher'' and the ``outer teacher'' clash, we usually opt for the inner one and trust our gut feeling; e.g.\ Jump and do a roll, but it doesn't feel safe, so you don't do it.
    \item \textbf{lala land}: Referring to anything which is more commonly used in the area of metaphysics, esoteric new age talk and superstitions, yet using it nevertheless due to lack of better words; e.g.\ when talking about reaching beyond the physical body.
\end{itemize}

\subsection{Zoo}\label{subsec:zoo}

These are names which refer to positions, techniques and ``qualities of body''.

% TODO finish animals
\begin{itemize}
    \item \textbf{little animal}: x
    \item \textbf{panda}: x
    \item \textbf{octopus}: x
    \item \textbf{elephant}: x
    \item \textbf{koala}: x
    \item \textbf{snake}: x
    \item \textbf{banana}: x
\end{itemize}
