\section{Jargon}\label{sec:jargon}

The following chapter is a rather class specific one, as the language and images developed in the scene ``I grew up in'' are very local and not universal at all.
Yet, they might inspire others, or at least provide some humorist benefit.

\subsection{Vocabulary}\label{subsec:vocabulary}

Sometimes concept are too complex, too difficult to verbalize them fully in their precise definition; or simply there have been no proper words established yet in our (English) language, which made it necessary to develop our own few words to convey a certain meaning efficiently:

\begin{itemize}
    \item \textbf{umpf}: The preferred quality of contact between two bodies which is characterized by properly sharing weight; touch is established throughout the whole body (-part) and there is a considerable amount of pressure; e.g.\ when both partners are in table top position and they feel firmly each other's groundedness.
    \item \textbf{wee}: A scream usually expressed only in moments of heightened levels of alertness/fear, to down-regulate one's own nervous system; also to simply express joy in the moment about a movement; e.g.\ when performing more risky lifts.
    \item \textbf{botsen}: A conflict which arises when the teacher's instructions lead to a resistance based on an internal wisdom; when the ``inner teacher'' and the ``outer teacher'' clash, we usually opt for the inner one and trust our gut feeling; e.g.\ Jump and do a roll, but it doesn't feel safe, so you don't do it.
    \item \textbf{lala land}: Referring to anything which is more commonly used in the area of metaphysics, esoteric new age talk and superstitions, yet using it nevertheless due to lack of better words; e.g.\ when talking about reaching beyond the physical body.
    \item \textbf{muschi muschi}: Indicating that during a dance a too sensual and thus inappropriate atmosphere arises.
    Usually when starting with flat dances, preceded by a body-scan, where things can even end up in a sort of ``cuddle puddle''.
\end{itemize}

\subsection{Zoo}\label{subsec:zoo}

These are names which refer to positions, techniques and ``qualities of body''.

\begin{itemize}
    \item \textbf{Bear}: Similar to the Koala, but while sitting on the ground and the bear hugging around the torso.
    \item \textbf{Banana}: Not really an animal, but anyway a useful metaphor of a way of movement on the ground, rolling sideways and only the core touching the floor; arms and legs stretched out long.
    \item \item \textbf{Little Animal}: A table-top position on all fours, yet emphasizing a more dynamic, alive quality than a regular, wooden table.
    \item \textbf{Elephant}: Although we do like elephants, but we don't like them in the dance studio, as their name is used to refer to steps, or landing of the feet, which make a loud sound, indicating that there was no control.
    Whenever we land silently, it is done so with control and elegance, which ultimately can prevent (serious) injuries.
    \item \textbf{Gorilla}: We differentiate between a ``good gorilla'' (hollow back) and a ``bad gorilla'' (arched back) which becomes important during lifting a partner.
    \item \textbf{Koala}: When doing a shoulder lift, the initial step could be hugging the upper body of the partner and being very close to his center.
    \item \textbf{Octopus}: A movement quality which indicates aliveness in all body parts, each of them being controlled by their own intelligence, fluid and smooth.
    % panda? snake?
\end{itemize}
